\documentclass[10pt, twoside, dvipsnames]{article}
\linespread {1.25}\selectfont
%\usepackage[utf8]{inputenc}
\usepackage{sectsty}
%\usepackage{german} //collides with the coloured boxes unfortunately
\usepackage[dvipsnames]{xcolor} %must be defined before tikz
\usepackage{CJKutf8}
\usepackage{ucs}
\usepackage{longtable}
\usepackage[encapsulated]{CJK}
\usepackage[colorlinks=true,
  urlcolor=ForestGreen,
  linkcolor=Blue,
  citecolor=Violet,
  pagebackref=true,
  pdftitle={Titel},
  pdfsubject={},
  pdfauthor={},
  pdfkeywords={keywords}
]{hyperref}
\usepackage{graphicx}
\usepackage{tabularx}
\usepackage{tabu}
\usepackage{nameref}
\usepackage{tikz-cd}
\usetikzlibrary{shapes.geometric}% für ellipse
\usepackage{amsmath} 	%MATHE1 (reicht für fast alles)
\usepackage{amssymb}	%MATHE2
\usepackage{amsthm}	%MATHE3
\usepackage{mathtools}    %MATHE4
\usepackage{bm}    %for italic fonts in mathmode by using e.g. $\bm{\mathcal{..}}$
\usepackage{float}
\usepackage{simplewick} 	%it works like
\usepackage{slashed} 	%For Dirac Notation
\usepackage{multicol}
\usepackage[headheight=16pt]{geometry}
\usepackage{fancyhdr}
\pagestyle{fancy}
\usepackage{ifthen} %To check if the page numbers are odd or not
\usepackage{framed,color}
\usepackage{enumitem}
\usepackage{mdwlist}
\usepackage{pdfpages}
\usepackage[arrow, matrix, curve]{xy}
\usepackage[most]{tcolorbox}
\DeclareMathAlphabet{\mathpzc}{OT1}{pzc}{m}{it}
\usepackage{ stmaryrd }
\SetSymbolFont{stmry}{bold}{U}{stmry}{m}{n} %necessary for being able to write $\bm{\mathcal{\ldots}}$
\usepackage{wrapfig}
\usepackage[toc,page]{appendix}
\usepackage{changepage}   % for the adjustwidth environment
\usepackage{cite} %for bibtex

\newenvironment{tab}{\begin{adjustwidth}{1cm}{}}{\end{adjustwidth}}
\dimen\footins12truein
\raggedbottom
\newtcolorbox{blueshadedcvbox}[1][]{enhanced jigsaw,
  colback=white!70!blue,
  coltext={black},
  boxrule=0pt,
  arc=4mm,
  auto outer arc,
  boxsep=7pt,
  left=4pt,
  right=2pt,
  bottom=2pt,
  top=2pt,
  %fontupper={\bfseries}, % other options: \sffamily , ...
  #1}
\newtcolorbox{greenshadedcvbox}[1][]{enhanced jigsaw,
  colback=white!60!green,
  coltext={black},
  boxrule=0pt,
  arc=4mm,
  auto outer arc,
  boxsep=7pt,
  left=4pt,
  right=2pt,
  bottom=2pt,
  top=2pt,
  %fontupper={\bfseries}, % other options: \sffamily , ...
  #1}
\newtcolorbox{yellowshadedcvbox}[1][]{enhanced jigsaw,
  colback=white!30!yellow,
  coltext={black},
  boxrule=0pt,
  arc=4mm,
  auto outer arc,
  boxsep=7pt,
  left=4pt,
  right=2pt,
  bottom=2pt,
  top=2pt,
  %fontupper={\bfseries}, % other options: \sffamily , ...
  #1}

%See ``new commands'' below in order to know how to use the above colorboxes

\geometry{
	a4paper,
	top=30mm,
	bottom=30mm,
	hmargin=18mm
	%twocolumn
}

\parindent 0pt
\def\wider{%
   \advance\leftskip -\parindent
   \advance\rightskip -\parindent}

\newcommand{\myfont}{gkai} % use  {gbsn} or {gkai} -- [And for traditional chinese: {bsmi}, {bkai}]

\newcommand{\fint}{\mathrel{\int \frac{d^3p}{(2\pi)^3} }}
\newcommand{\cor}{\mathrel{\widehat{=}}}
\newcommand{\Ra}{\mathrel{ \Rightarrow }}
\newcommand{\La}{\mathrel{ \Leftarrow }}
\newcommand{\ra}{\mathrel{ \rightarrow }}
\newcommand{\Lr}{\mathrel{ \Leftrightarrow }}
\newcommand{\lr}{\mathrel{ \leftrightarrow }}
\newcommand{\nn}{\mathrel{ \nonumber  }}
\newcommand{\pa}{\mathrel{ \partial}}
\newcommand{\te}[1]{\text{#1}}
\newcommand{\tf}[1]{\textbf{#1}}
\newcommand{\ti}[1]{\textit{#1}}
\newcommand{\ov}[1]{\mathrel{ \overset{ \text{(#1)} }{=}  }}
\newcommand{\br}[1]{\mathrel{ \left(#1\right)  }}
\newcommand{\Br}[1]{\mathrel{ \left[#1\right]  }}
\newcommand{\bbr}[1]{\mathrel{ \left\{#1\right\}  }}

\newcommand{\at}[1]{\mathrel{ \left.#1\right| }}
\newcommand{\bra}[1]{\mathrel{ \left\langle #1 \right|  }}
\newcommand{\ket}[1]{\mathrel{ \left| #1 \right\rangle  }}
\newcommand{\un}[2]{\mathrel{ \underbrace{#1}_{#2}  }}
\newcommand{\mc}[1]{\begin{multicols}{2} #1 \end{multicols} }
\newcommand{\id}{\ensuremath{\mathbbm{1}}}
\newcommand{\s}[1]{\sage{#1}}
\newcommand{\p}[2]{\sageplot[scale=.#1]{#2}}
\newcommand{\cs}[1]{\begin{sagesilent} #1 \end{sagesilent} }
\newcommand{\bb}[1]{\begin{blueshadedcvbox}#1\end{blueshadedcvbox}}
\newcommand{\gb}[1]{\begin{greenshadedcvbox}#1\end{greenshadedcvbox}}
\newcommand{\yb}[1]{\begin{yellowshadedcvbox}#1\end{yellowshadedcvbox}}
\newcommand{\be}{\begin{enumerate}[label=(\alph*)]}
\newcommand{\re}{\resume{enumerate}[{[label=(\alph*)]}]}
\newcommand{\se}{\suspend{enumerate}}
\newcommand{\ee}{\end{enumerate}}
\newcommand{\mf}[1]{\mrahmen[rounded corners]{mfarbe}{#1}}
\newcommand{\red}[1]{\color{red}#1 \color{black}}
\newcommand{\blue}[1]{\color{blue}#1 \color{black} }
\newcommand{\green}[1]{\color{green}#1 \color{black} }
\newcommand{\yel}[1]{\color{yellow}#1 \color{black} }
\newcommand{\bo}[1]{\bold{#1}}
\newcommand{\eck}{\lrcorner}
\newcommand{\vp}{\varphi}
\newcommand{\pr}{\textbf{Proof}}
\newcommand{\cat}[1]{\bm{\mathcal{#1}}}
\newcommand{\fun}[1]{\mathcal{#1}}
\newcommand{\vn}{\varnothing}
\newcommand{\mm}[2]{\begin{pmatrix}\text{#1}\\\text{#2}\end{pmatrix}}
\newcommand{\ty}[1]{$(\mathcal{P}_{#1},\bm{\mathcal{C}}_{#1})$}

\newcommand{\bt}[1][normal]{\begin{tikzcd}[ampersand replacement = \&, column sep=#1,row sep=#1]}
%other options are: tiny, small, scriptsize, normal, large, huge
\newcommand{\et}{\end{tikzcd}}

\long\def\/*#1*/{}
\/* Finally a command
that can define multiple line comments
even for equations! = ) */
\iffalse This
is an
alternative \fi
\def\myline{
 \noindent\makebox[\linewidth]{\rule{\textwidth}{.4pt}}  % 0.4pt is the default thickness
}

\definecolor{shadecolor}{rgb}{0,0,1}
%\pagestyle{headings}
%\copypagestyle{chapter}{plain} % make chapter/section a page style of its own

\pagestyle{fancy}
\fancyhf{}

\fancyheadoffset[LE,RO]{0mm
%\marginparsep+\marginparwidth
}

\renewcommand{\contentsname}{\large Contents}
\renewcommand{\headrulewidth}{0pt}
\renewcommand{\footrulewidth}{0pt}
\rhead{}
\lhead{}
\cfoot{~~~~~~~~~~~ \thepage }

\setlength{\columnsep}{1.1cm}
\setlength{\jot}{5px} %set the equation line distance

