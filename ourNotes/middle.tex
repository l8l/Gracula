\section{Generalisations of the circle}

Here is a short summary of some ideas outlined in \cite{Hanson1994}.

The equation of the circle is given by
\begin{equation}
  x^2+y^2 = 1 \te{ with solutions }
  \begin{pmatrix}
    x\\
    y
  \end{pmatrix} =
  \begin{pmatrix}
    \cos \theta\\
    \sin \theta
  \end{pmatrix},~\theta \in [0,2\pi).
  \label{eq:circle}
\end{equation}
A generalisation is given by
\begin{equation}
  x^n+y^n = 1 \te{ with solutions }
  \begin{pmatrix}
    x\\
    y
  \end{pmatrix} =
  \begin{pmatrix}
    (\cos \theta)^{2/n}\\
  (\sin \theta)^{2/n}
\end{pmatrix}.
  \label{eq:generalRealCircle}
\end{equation}
This equation is already interesting for 2D graphic visualisations because
the solutions for different values of $n\ge 2$ correspond to interpolations between
the circle and and a square with rounded edges while $n=1$ is a perfect square.

One may define a more general equation on the complex plane,
\begin{equation}
  u^2+o^2 = 1,~u,o\in\mathbb{C},
\end{equation}
which is solved by
\begin{equation}
  \begin{split}
    u = \frac{1}{2}\br{e^{\xi+i\theta}+e^{-(\xi+i\theta)}},\qquad
  o = \frac{1}{2i}\br{e^{\xi+i\theta}-e^{-(\xi+i\theta)}}
\end{split},~\xi,\theta\in\mathbb{R}
  \label{eq:complexCircle}
\end{equation}
because
\begin{equation*}
  \begin{split}
    u^2 = \frac{1}{4}
    \br{e^{2(\xi+i\theta)}+e^{-2(\xi+i\theta)}+2} \te{ and }
    o^2 = -\frac{1}{4}
    \br{e^{2(\xi+i\theta)}+e^{-2(\xi+i\theta)}-2}
  \end{split}.
\end{equation*}
Observe that we now actually solve two equations
(because $\mathbb{C}\simeq \mathbb{R}^2$ and if we use basis $(1,i)$, then one complex equation
amounts to two real equations) for two complex variables (4 real components)
and thus must have 2 free real parameters in the end which amount to $\xi,\theta\in \mathbb{R}$.
Thus, the solution describes a 2-dimensional surface.\\

The real part of the solution is
\begin{equation}
  \begin{split}
    \Re{u} = \frac{e^{\xi}+e^{-\xi}}{2}\cos\theta,\qquad
    \Re o = \frac{e^{\xi}+e^{-\xi}}{2}\sin\theta
\end{split}
\end{equation}
which restricts back to a solution of eq. (\ref{eq:circle}) for $\xi=0$.\\
However, by defining $s(k,n):=\exp\br{2\pi i k/n}$, we can solve the more general equation
\begin{equation}
  \begin{split}
    z_1^{n_1}+z_2^{n_2}=1\qquad \te{ with } \qquad z_1 = s(k_1,n_1)u^{2/n_1},~z_2 = s(k_2,n_1)o^{2/n_2},
\end{split}
  \label{eq:complexGeneralCircle}
\end{equation}
%the real part of which also restricts back to solutions of eq. (\ref{eq:generalRealCircle}) for $\xi=0$
%and $n_1=n_2$.
which gives rise to the visualisation of a multitude of interesting 2D - surfaces embedded in 3-space.

